\documentclass[10pt]{article}

\usepackage[T1]{fontenc}
\usepackage{geometry}
\usepackage{amsmath, amssymb, amsthm}

\title{Mathematics I - Problem Sheet VIII}
\author{Satvik Saha}
\date{}

\geometry{a4paper, margin=1in}
\setlength\parindent{0pt}
\renewcommand{\labelenumi}{(\roman{enumi})}
% \renewcommand\qedsymbol{$\blacksquare$}

\begin{document}
	\par\textbf{IISER Kolkata} \hfill \textbf{Problem Sheet VIII}
	\vspace{3pt}
	\hrule
	\vspace{3pt}
	\begin{center}
		\LARGE{\textbf{MA 1101 : Mathematics I}}
	\end{center}
	\vspace{3pt}
	\hrule
	\vspace{3pt}
	Priyanshu Mahato, \texttt{21MS002}\hfill\today
	\vspace{20pt}
	
	\textbf{Solution 1.}
	Let $\emptyset \neq D \subseteq \mathbb{R}$, let $c \in D$ and let $f\colon D \to \mathbb{R}$ be continuous at $c$ with $f(c) > 0$.
	We claim that there exists $\delta > 0$ such that \[f(x) > 0, \text{ for all } x \in (c - \delta, c + \delta) \cap D.\]
	
	Since $f$ is continuous at $c$, we find $\delta_c > 0$ such that
	\[|f(x) - f(c)| < \frac{1}{2}f(c), \text{ for all } x \in (c - \delta_c, c + \delta_c) \cap D.\]
	
	Suppose that our claim is false, i.e.\ there exists at least one $x_0 \in (c - \delta_c, c + \delta_c) \cap D$ such that $f(x_0) \le 0$.
	Then, $f(x_0) - f(c) < 0 \Rightarrow |f(x_0) - f(c)| = f(c) - f(x_0) \ge f(c)$, a contradiction.
	Hence, setting $\delta = \delta_c$ proves our claim. \qed\\
	
	\textbf{Solution 2.}
	Let $\emptyset \neq D \subseteq \mathbb{R}$, let $c \in D$ and let $f, g\colon D \to R$ be continuous at $c$.
	\begin{enumerate}
		\item We claim that $f + g$ is continuous at $c$.
		
		Let $\epsilon > 0$ be given.
		We find $\delta_f, \delta_g$ such that for all $x \in D$,
		\[|x - c| < \delta_f \implies |f(x) - f(c)| < \epsilon/2,\]
		\[|x - c| < \delta_g \implies |g(x) - g(c)| < \epsilon/2.\]
		We set $\delta = \min\{\delta_f, \delta_g\}$. Then, for all $x \in D$ satisfying
		$|x - c| < \delta$, we have
		\begin{align*}
			|(f(x) + g(x)) - (f(c) + g(c))| \;&=\; |(f(x) - f(c)) + (g(x) - g(c))| \\
			\;&\le\; |f(x) - f(c)| + |g(x) - g(c)| \\
			\;&<\; \epsilon/2 + \epsilon/2 \\
			\;&=\; \epsilon
		\end{align*}
		This proves our claim. \qed
		
		\item We claim that for all $\alpha \in \mathbb{R}$, $\alpha f$ is continuous at $c$.
		
		Let $\epsilon > 0$ be given.
		If $\alpha \neq 0$, we find $\delta_f$ such that for all $x \in D$,
		\[|x - c| < \delta_f \implies |f(x) - f(c)| < \epsilon/|\alpha|.\]
		We set $\delta = \delta_f$. Then, for all $x \in D$ satisfying
		$|x - c| < \delta$, we have
		\begin{align*}
			|\alpha f(x) - \alpha f(c)| \;&=\; |\alpha| |f(x) - f(c)| \\
			\;&<\; |\alpha| \frac{\epsilon}{|\alpha|} \\
			\;&=\; \epsilon
		\end{align*}
		
		If $\alpha = 0$, we trivially have
		\[|x - c| < \delta = \epsilon \implies |\alpha f(x) - \alpha f(c)| = 0 < \epsilon.\]
		
		This proves our claim. \qed
		
		\item We claim that $fg$ is continuous at $c$.
		
		Let $\epsilon > 0$ be given.
		We find $\delta_1, \delta_2, \delta_3, \delta_4$ such that for all $x \in D$,
		\[|x - c| < \delta_1 \implies |f(x) - f(c)| < \sqrt{\epsilon/2},\]
		\[|x - c| < \delta_2 \implies |g(x) - g(c)| < \sqrt{\epsilon/2},\]
		\[|x - c| < \delta_3 \implies |f(x) - f(c)| < \epsilon/4(1 + |g(c)|),\]
		\[|x - c| < \delta_4 \implies |g(x) - g(c)| < \epsilon/4(1 + |f(c)|).\]
		
		We set $\delta = \min\{\delta_1, \delta_2, \delta_3, \delta_4\}$.
		Then, for all $x \in D$ satisfying $|x - c| < \delta$, we have
		\begin{align*}
			|(fg)(x) - (fg)(c)| \;&=\; |f(x)g(x) - f(c)g(c)| \\
			\;&=\; |(f(x) - f(c) + f(c))(g(x) - g(c) + g(c)) - f(c)g(c)| \\
			\;&=\; |(f(x) - f(c))(g(x) - g(c)) + f(c)(g(x) - g(c)) + g(c)(f(x) - f(c)) + f(c)g(c) - f(c)g(c)| \\
			\;&=\; |(f(x) - f(c))(g(x) - g(c)) + f(c)(g(x) - g(c)) + g(c)(f(x) - f(c))| \\
			\;&\le\; |f(x) - f(c)| |g(x) - g(c)| + |f(c)| |g(x) - g(c)| + |g(c)| |f(x) - f(c)| \\
			\;&<\; \sqrt\frac{\epsilon}{2} \sqrt\frac{\epsilon}{2} + \frac{|f(c)|\epsilon}{4(1 + |f(c|)} 
			+ \frac{|g(c)|\epsilon}{4(1 + |g(c|)} \\
			\;&<\; \frac{\epsilon}{2} + \frac{\epsilon}{4} + \frac{\epsilon}{4} \\
			\;&=\; \epsilon
		\end{align*}
		
		This proves our claim. \qed
		
		\item We claim that if $g(c) \neq 0$, $f /g$ is continuous at $c$.
		To prove this, we first show that $h\colon D \to \mathbb{R}$, $h(x) = 1/g(x)$ is
		continuous at $c$.
		
		Let $\epsilon > 0$ be given.
		We find $\delta_1, \delta_2$ such that for all $x \in D$,
		\[|x - c| < \delta_1 \implies |g(x) - g(c)| < \frac{1}{2}|g(c)|,\]
		\[|x - c| < \delta_2 \implies |g(x) - g(c)| < \frac{1}{2}\epsilon |g(c)|^2.\]
		
		We set $\delta = \min\{\delta_1, \delta_2\}$.
		Then, for all $x \in D$ satisfying $|x - c| < \delta$, we have
		\begin{align*}
			\frac{1}{2}|g(c)| \;&>\; |g(x) - g(c)| \\
			\;&\ge\; | |g(x)| - |g(c)| | \\
			\;&\ge\; |g(c)| - |g(x)| \\
			|g(x)| \;&>\; \frac{1}{2}|g(c)| > 0\\
			\frac{1}{|g(x)|} \;&<\; \frac{2}{|g(c)|} \\
			\left| h(x) - h(c) \right|
			\;&=\; \left| \frac{1}{g(x)} - \frac{1}{g(c)} \right| \\
			\;&=\; \frac{|g(x) - g(c)|}{|g(c)g(x)|} \\
			\;&=\; |g(x) - g(c)| \frac{1}{|g(c)| |g(x)|} \\
			\;&<\;  \frac{1}{2}\epsilon|g(c)|^2 \frac{2}{|g(c)|^2} \\
			\;&=\; \epsilon
		\end{align*}
		Thus, $h$ is continuous at $c$.
		Therefore, $f /g = fh$ is continuous at $c$. \qed
	\end{enumerate}
	
	\textbf{Solution 3.}
	Let $I \subseteq \mathbb{R}$ be an open interval, let $c \in I$ and let $f, g\colon D \to R$ be differentiable at $c$.
	
	Note that $f, g$ are continuous at $c$.
	
	Since $f, g$ are differentiable at $c$, we have the following.
	\[f'(c) \;=\; \lim_{x \to c} \frac{f(x) - f(c)}{x - c}\]
	\[g'(c) \;=\; \lim_{x \to c} \frac{g(x) - g(c)}{x - c}\]
	% Note that $c$ is a limit point of $I$.
	\begin{enumerate}
		\item We claim that $f + g$ is differentiable at $c$ and $(f + g)'(c) = f'(c) + g'(c)$.
		
		Note that
		\begin{align*}
			f'(c) + g'(c) \;&=\; \lim_{x \to c} \frac{f(x) - f(c)}{x - c} + \lim_{x \to c} \frac{g(x) - g(c)}{x - c} \\
			\;&=\; \lim_{x \to c} \frac{f(x) - f(c)}{x - c} + \frac{g(x) - g(c)}{x - c} \\
			\;&=\; \lim_{x \to c} \frac{(f(x) + g(x)) - (f(c) + g(c))}{x - c} \\
			\;&=\; \lim_{x \to c} \frac{(f + g)(x) - (f + g)(c)}{x - c} \\
			\;&=\; (f + g)'(c)
		\end{align*}
		Hence,
		\[
		(f + g)'(c) \;=\; \lim_{x \to c} \frac{(f + g)(x) - (f + g)(c)}{x - c} \;=\; f'(c) + g'(c)
		\]
		This proves our claim. \qed
		
		\item We claim that for all $\alpha \in \mathbb{R}$, $\alpha f$ is differentiable at $c$ and $(\alpha f)'(c) = \alpha f'(c)$.
		
		Note that
		\begin{align*}
			\alpha f'(c) \;&=\; \alpha \lim_{x \to c} \frac{f(x) - f(c)}{x - c} \\
			\;&=\; \lim_{x \to c} \frac{\alpha f(x) - \alpha f(c)}{x - c} \\
			\;&=\; \lim_{x \to c} \frac{(\alpha f)(x) - (\alpha f)(c)}{x - c} \\
			\;&=\; (\alpha f)'(c)
		\end{align*}
		Hence,
		\[
		(\alpha f)'(c) \;=\; \lim_{x \to c} \frac{(\alpha f)(x) - (\alpha f)(c)}{x - c} \;=\; \alpha f'(c)
		\]
		
		This proves our claim. \qed
		
		\item We claim that $fg$ is differentiable at $c$ and $(fg)'(c) = f'(c)g(c) + f(c)g'(c)$.
		
		Note that since $c$ is a limit point of $I$, $f(c) = \lim_{x \to c} f(x)$ and $g(c) = \lim_{x \to c} g(x)$.
		\begin{align*}
			f'(c)g(c) + f(c)g'(c)
			\;&=\; g(c) \lim_{x \to c} \frac{f(x) - f(c)}{x - c} + \lim_{x \to c} f(x) \lim_{x \to c} \frac{g(x) - g(c)}{x - c} \\
			\;&=\; \lim_{x \to c} \frac{(f(x) - f(c))g(c) + f(x)(g(x) - g(c))}{x - c} \\
			\;&=\; \lim_{x \to c} \frac{f(x)g(c) - f(c)g(c) + f(x)g(x) - f(x)g(c)}{x - c} \\
			\;&=\; \lim_{x \to c} \frac{f(x)g(x) - f(c)g(c)}{x - c} \\
			\;&=\; \lim_{x \to c} \frac{(fg)(x) - (fg)(c)}{x - c} \\
			\;&=\; (fg)'(c)
		\end{align*}
		
		Hence,
		\[
		(fg)'(c) \;=\; \lim_{x \to c} \frac{(fg)(x) - (fg)(c)}{x - c} \;=\; f'(c)g(c) + f(c)g'(c)
		\]
		
		This proves our claim. \qed
		
		
		\item We claim that if $g(c) \neq 0$, $f /g$ is differentiable at $c$ and $(f /g)'(c) = (f'(c)g(c) - f(c)g'(c))/g(c)^2$.
		To prove this, we first show that $h\colon D \to \mathbb{R}$, $h(x) = 1/g(x)$ is
		differentiable at $c$ and $h'(c) = -g'(c) /g(c)^2$.
		
		Note that $h$ is continuous and $c$ is a limit point of $I$, hence $h(c) = \lim_{x \to c} h(x)$.
		\begin{align*}
			-\frac{g'(c)}{g(c)^2} \;&=\; -\frac{1}{g(c)^2} \lim_{x \to c} \frac{g(x) - g(c)}{x - c} \\
			\;&=\; \frac{1}{g(c)} \lim_{x \to c}\frac{1}{g(x)} \lim_{x \to c} \frac{g(c) - g(x)}{x - c} \\
			\;&=\; \lim_{x \to c} \frac{\frac{1}{g(x)} - \frac{1}{g(c)}}{x - c} \\
			\;&=\; \lim_{x \to c} \frac{h(x) - h(c)}{x - c} \\
			\;&=\; h'(c)
		\end{align*}
		
		Hence,
		\[
		h'(c) \;=\; -g'(c)/g(c)^2
		\]
		Using the product rule,
		\[
		(f /g)'(c) \;=\; (fh)'(c) \;=\; f'(c)h(c) + f(c)h'(c) \;=\; f'(c)/g(c) - f(c)g'(c)/g(c)^2
		\]
		
		This proves our claim. \qed
	\end{enumerate}
	
	\textbf{Solution 4.}
	\begin{enumerate}
		\item We claim that for all $x > 0$,
		\[ \frac{x}{1 + x} < \ln(1 + x) < x. \]
		
		Let $f, g\colon (0, \infty) \to \mathbb{R}$ be defined as follows.
		\[f(x) \;=\; \ln(1 + x) - \frac{x}{1 + x}, \text{ for all } x > 0,\]
		\[g(x) \;=\; x - \ln(1 + x), \text{ for all } x > 0,\]
		
		We note that
		\begin{align*}
			f'(x) \;&=\; \frac{1}{1 + x} - \frac{(1 + x) - x}{(1 + x)^2} \\
			\;&=\; \frac{(1 + x) - (1 + x) + x}{(1 + x)^2} \\
			\;&=\; \frac{x}{(1 + x)^2} \\
			\;&>\; 0 \\
			g'(x) \;&=\; 1 - \frac{1}{1 + x} \\
			\;&=\; \frac{(1 + x) - 1}{1 + x} \\
			\;&=\; \frac{x}{1 + x} \\
			\;&>\; 0
		\end{align*}
		
		Thus, $f$ and $g$ are monotonically increasing on $(0, \infty)$. We can write
		\begin{align*}
			f(x) > \lim_{t \to 0} f(t) = 0 \\
			g(x) > \lim_{t \to 0} g(t) = 0
		\end{align*}
		Therefore, 
		\[\ln(1 + x) > \frac{x}{1 + x}\]
		\[x > \ln(1 + x)\]
		
		This proves our claim. \qed
		
		\item We claim that for all $x > 0$,
		\[ e^x > 1 + x + \frac{1}{2}x^2. \]
		
		Let $f\colon [0, x] \to \mathbb{R}$ be defined as $f(t) = e^t$, for all $t \in [0, x]$.
		Clearly, $f$ is continuous in $[0, x]$ and differentiable in $(0, x)$. Note that $f'(t) = f(t) = e^t$.
		Hence, $f, f'$ are continuous on $[0, x]$ and $f'' = f$ exists in $(0, x)$.
		
		Using Taylor's Theorem, we find $c \in (0, x)$ such that
		\[e^x = e^0 + e^0(x - 0) + \frac{1}{2}e^c(x - 0)^2.\]
		
		Since, $e^0 = 1$ and $e^c > 1$ for $c > 0$, we have
		\[e^x > 1 + x + \frac{1}{2}x^2.\]
		
		This proves our claim. \qed
		
		
		\item We claim that for all $x, y \in \mathbb{R}$,
		\[
		|\sin{x} - \sin{y}| \le |x - y|.
		\]
		
		Note that if $x = y$, our claim is trivially true.
		
		Without loss of generality, let $x > y$.
		Let $f, g\colon [x, y] \to \mathbb{R}$ be defined as follows.
		\[ f(t) \;=\; \sin{t}, \text{ for all } t \in [x, y],\]
		\[ g(t) \;=\; t, \text{ for all } t \in [x, y].\]
		
		Clearly, $f$ and $g$ are continuous in $[x, y]$ and differentiable in $(x, y)$.
		Note that $f'(t) = \cos{t}$ and $g'(t) = 1$.
		
		Using Cauchy's Mean Value Theorem, we find $c \in (x, y)$ such that.
		\[(\sin{x} - \sin{y}) \;=\; (x - y)\cos{c}.\]
		
		Since $\cos{c} \le 1$,
		\[|\sin{x} - \sin{y}| \le |x - y|.\]
		
		This proves our claim. \qed
	\end{enumerate}
\end{document}