\documentclass[11pt]{scrartcl}
\usepackage{answers}
\Newassociation{sketch}{hintitem}{hints}
\renewcommand{\solutionextension}{out}
\usepackage[none]{hyphenat}
%\usepackage[margin=1in]{geometry}

\usepackage[sexy]{evan}
\newcommand\EE{\mathbb E}
\newcommand\PP{\mathbb P}
%\setlength{\parindent}{0em}
%\setlength{\parskip}{2.5em}

\begin{document}
	\title{Module 2: Number Systems}
	\author{Priyanshu Mahato}
	\date{\today}
	\maketitle
	
	\begin{abstract}
		Email: \mailto{pm21ms002@iiserkol.ac.in}.These are my personal notes on Number Systems. We will consider $\mathbb{N}$ (Natural Numbers), $\mathbb{Z}$ (Integers), and $\mathbb{Q}$ (rational Numbers), but not $\mathbb{R}$ (Real Numbers).
	\end{abstract}

	\section{Natural Numbers}
	
		The natural numbers are $1,2,3,4,\dots$. The set of all natural numbers is denoted by $\NN$.
		
		\begin{definition}
			We assume familiarity with the algebraic operations of addition and multiplication on the set $\NN$ and also with the linear order relation $<$ on $\NN$ defined by ``$a<b$ if $a, b \in \NN$ and $a$ is less than $b$".
		\end{definition}
	
		We discuss the following fundamental properties of the set $\NN$.
		
		\begin{enumerate}
			\item Well Ordering Property
			\item  Principle of Induction
		\end{enumerate}
	
	\subsection{Well Ordering Property}
		\begin{definition}
			Every non-empty subset of $\NN$ has a least element.
		\end{definition}
		This means that if $S$ is a non-empty subset of $\NN$, then there is an element $m$ in $S$ such that $m \leq s$ for all $s \in S$.
		
		In particular, $\NN$ itself has the least element 1.
		
		\begin{proof}
			Let $S$ be a non-empty subset of $\NN$. Let $k$ be an element of $S$. Then $k$ is a natural number.
			
			We define a subset $T$ by $T = \left\{ x \in S : x \leq k \right\}$. The $T$ is a non-empty subset of $\left\{ 1,2,3,\dots, k \right\}$.
			
			Clearly, $T$ is a finite subset of $\NN$ and therefore it has a least element, say $m$. Then $1 \leq m \leq k$.
			
			We now show that $m$ is the least element of $S$. Let $s$ be any element of $S$.
			
			If $s>k$, then the inequality $m \leq k$ implies $m<s$.
			
			If $s \leq k$, the $s \in T$; and $m$ being the least element of $T$, we have $m \leq s$.
			
			Thus $m$ is the least element of $S$.
		\end{proof}
	
	\subsection{Principle of Induction}
	\begin{definition}
		Let $S$ be a subset of $\NN$ such that,
		\begin{enumerate}[i)]
			\item  $1 \in S$ and,
			\item  if $k \in S$, then $k+1 \in S$.
		\end{enumerate}
	Then $S = \NN$
	\end{definition}
	
	\begin{proof}
		Let $T = \NN - S$. We prove that $T = \phi$.
		
		Let $T$ be non-empty. then by the \emph{Well Ordering Property} of $\NN$, the non-empty subset $T$ has a least element, say $m$.
		
		Since $1 \in S$ and $1$ is the least element of $\NN$, $m>1$.
		
		Hence, $m-1$ is a natural number and $m-1 \notin T$. So, $m-1 \in S$.
		
		But by ii) $m-1 \in S \Rightarrow (m-1)+1 \in S$, i.e., $m \in S$.
		
		This contradicts that $m$ is the least element in $T$. Therefore, our assumption is wrong and $T = \phi$.
		
		Therefore, $S=\NN$.
	\end{proof}
		 
	\begin{theorem}
		Let $P(n)$ be a statement involving a natural number $n$. If,
		\begin{enumerate}[i)]
			\item $P(1)$ is true, and
			\item $P(k+1)$ is true whenever $P(k)$ is true,
		\end{enumerate}
		then $P(n)$ is true for all $n \in \NN$.
	\end{theorem}

	\begin{proof}
		Let $S$ be the set of those natural numbers for which the statement $P(n)$ is true.
		
		Then $S$ has the properties,
		\begin{enumerate}[(a)]
			\item $1 \in S$, by (i)
			\item $k \in S \Rightarrow k+1 \in S$ by (ii).
		\end{enumerate}
	
		By the \emph{Principle of Induction}, $S = \NN$.
		
		Therefore, $P(n)$ is true for all $n \in \NN$.
	\end{proof}

	\begin{remark}
		Let a statement $P(n)$ satisfies the conditions,
		\begin{enumerate}[(i)]
			\item for some $m \in \NN,\ P(m)$ is true ($m$ being the least possible); and
			\item $P(k)$ is true $\Rightarrow$ $P(k+1)$ is true for all $k \geq m$.
		\end{enumerate}
		Then $P(n)$ is true for all natural numbers $\geq m$.
	\end{remark}

	\pagebreak

	\begin{flushleft}
			\textbf{Worked Examples}
	\end{flushleft}
	\begin{example}
		Prove that for each $n \in \NN$, $1+2+3+4+\dots +n = \frac{n(n+1)}{2}$ for all $n \in \NN$.
	\end{example}

	The statement is true for $n=1$, because $1 = \dfrac{1(1+1)}{2}$.\\
	Let the statement be true for some natural number $k$.\\
	Then $1+2+3+4+\dots +k = \dfrac{(k+1)}{2}$ and therefore,\\
	$(1+2+\dots+k)+(k+1) = \dfrac{k(k+1)}{2} + (k+1)$\\
	or, $1 +2 +3 + \dots +(k+1) = \dfrac{(k+1)(k+2)}{2}$.
	
	This shows that the statement is true for the natural number $k+1$ if it is true for k. By the principle of induction, the statement is true for all natural numbers.
	
	\begin{example}
		Prove that for each $n \geq 2,\ (n+1)! > 2^{n}$.
	\end{example}

	The equality holds for $n=2$ since $(2+1)! > 2^{2}$.
	
	Let the inequality hold for some natural number $k \geq 2$.
	
	Then, $(k+1)! > 2^{k}$
	\begin{align*}
		and\ (k+2)! &= (k+2)(k+1)!\\
		&> 2 \cdot 2^{k}, since\ k+2>2\\
		or,\ (k+2)! &> 2^{k+1}
	\end{align*}

	This shows that if the inequality holds for $k(\geq 2)$ then it also holds for $k+1$.
	
	By the principle of induction, the inequality holds for all natural numbers $\geq 2$.
	
	[Note that the inequality does not hold for $n=1$.]
	
	\subsection{Second Principle of Induction (or, Principle of Strong Induction)}
	\begin{definition}
		Let $S$ be a subset of $\NN$ such that
		\begin{enumerate}[(i)]
			\item $1 \in S$, and
			\item if $\{1,2,3,4,\dots\} \subset S$, then $k+1 \in S$.
		\end{enumerate}
		Then $S = \NN$
	\end{definition}

	\begin{proof}
		Let $T = \NN - S$. We prove that $T = \phi$.
		
		Let $T$ be non-empty. Then $T$ will have a least element, say $m$, by the WOP of $\NN$. Since, $1 \in S,\ 1 \notin T$.
		
		As $m$ is the least element in $T$ and $1 \notin T$, $m>1$.
		
		By choice of $m$, all natural numbers less than $m$ belong to $S$. That is $1,2,\dots, m-1$ all belong to $S$.
		
		Then by (ii) $m \in S$ and consequently, $m \notin T$, a contradiction. It follows that $T = \phi$ and therefore, $S=\NN$.
	\end{proof}

\pagebreak

	\begin{flushleft}
			\textbf{Worked Examples}(continued)
	\end{flushleft}
	
	\begin{example}
		Prove that for all $n \in \NN$, $(3+\sqrt{5})^{n} + (3-\sqrt{5})^{n}$ is an even integer.
	\end{example}
	Let $P(n)$ be the statement ``$(3+\sqrt{5})^{n} + (3-\sqrt{5})^{n}$ is an even integer".
	
	$P(1)$ is true since $(3+\sqrt{5})^{1} + (3-\sqrt{5})^{1} = 6$, an even integer.
	
	Let us assume that $P(n)$ is true for $n=1,2,\dots,k$.
		\begin{align*}
			&(3+\sqrt{5})^{(k+1)} + (3-\sqrt{5})^{(k+1)}\\
			&= a^{(k+1)} + b^{(k+1)}\ where\ a=3+\sqrt{5},\ b=3-\sqrt{5}\\
			&= (a^{k} + b^{k})(a+b) - (a^{k-1} + b^{k-1})ab\\
			&= 6(a^{k} + b^{k}) - 4(a^{k-1} + b^{k-1}).		
		\end{align*}
	
	It is an even integer, since $a^{k}+b^{k}$ and $a^{k-1}+b^{k-1}$ are even integers.
	
	Hence, $P(k+1)$ is true whenever $P(n)$ is true for all $n=1,2,\dots,k$.
	
	By the second principle of induction, $P(n)$ is true for all natural numbers.
	
	\section{Integers}
	
	We shall now construct the set of integers using the set of Natural Numbers. Our construction will be through an equivalence relation on $\NN \times \NN$.
	
	\begin{definition}
		Define $\sim_{\ZZ}$ on $\NN \times \NN$ by, for all $(m,n),(p,q) \in \NN \times \NN$,
		 $$(m,n)\sim_{\ZZ}(p,q) \Leftrightarrow m+q = n+p$$
	\end{definition}

	\begin{lemma}
		\begin{enumerate}[i)]
			\item $\sim_{\ZZ}$ is an equivalence relation on $\NN \times \NN$.
			\item for all $(m,n) \in \NN \times \NN$,
				\[(m,n) \sim_{\ZZ}
				\begin{cases*}
					$(m+1-n, 1)$ & for $m\geq n$\\
					$(1, n+1-m)$ & for $n\geq m$
				\end{cases*}\]
			\item $\NN \times \NN / \sim_{\ZZ} = \{[(j,1)] : j \in \NN\ \&\ j\geq 2\} \cup \{[(1,k)] : k \in \NN\ \&\ k\geq 2\} \cup \{[(1,1)]\}$
		\end{enumerate}
	\end{lemma}

	\begin{proof}
		Look at MA1101 ps2 (\textbf{Problem 2})
	\end{proof}

	\begin{definition}
		Let us write $\ZZ \defeq \NN \times \NN / \sim_{\ZZ} = \{[(m,n)] : (m,n) \in \NN \times \NN\}$
		
		We also write,
		$$\boxed{\overline{0} \defeq [(1,1)]\ \&\ \overline{1} \defeq [(2,1)]}$$
		
		Let $\overline{a} \defeq [(m,n)]$, $\overline{b} \defeq [(p,q)] \in \ZZ$.
		\begin{enumerate}[i)]
			\item \textbf{Addition}\\
			$$\boxed{ a + b \defeq [(m+p, n+q)]}$$
			\item \textbf{Multiplication}\\
			$$\boxed{ a \cdot b \defeq [(mp+nq, mq+np)]}$$
		\end{enumerate}
	\label{definition:bar}
	\end{definition}

	We have the following important theorem,
	
	\begin{theorem}
		\begin{enumerate}[i)]
			\item $+$ is well-defined, commutative and associative
			\item $a+\overline{0} = a = \overline{0} + a$, $\forall a \in \ZZ$
			\item $\forall a \in \ZZ, \exists$ a unique $x \in \ZZ$, such that $a + x = \overline{0}$. We write $-a$ for $x$ and say that $-a$  is  the negative of $a$
			\item $\forall a,b \in \ZZ, \exists$ a unique $x \in \ZZ$ such that $a+x = b$
			\item $\cdot$ is well defined, associative, and commutative
			\item $a\cdot\overline{1} = a = \overline{1}\cdot a$, $\forall a \in \ZZ$
			\item $\forall a,b,c \in \ZZ$, $a\cdot (b+c) = a\cdot b + a\cdot c$
		\end{enumerate}
	\label{theorem:ring}
	\end{theorem}

	\begin{remark}
		In other words, we can call $(\ZZ, +, \cdot)$ as a commutative \textbf{ring} with identity.
	\end{remark}

	To prove \ref{theorem:ring}, we start off with a lemma,
	
	\begin{lemma}
		$\forall\ n,p,q \in \NN$, if $n+p = n+q$ $\Rightarrow p=q$
		\label{lemma:npq}
	\end{lemma}

	\begin{proof}
		We prove using Induction on $n$. When $n=1$, $S(p) = 1+p = 1+q = S(q)$. As $S$ is one-one (injective), it follows that $p=q$. Let us suppose that for some $k \in \NN$, $k+p = k+q$. Hence, $(k+1)+p = 1+(k+p) = 1+(k+q) = (k+1)+q$, i.e., the result holds when $n=k+1$. Therefore, the result is proved using Induction.
	\end{proof}

	We are now ready to prove Theorem \ref{theorem:ring},
	
	\begin{proof}
		i) We first check that $+$ is well-defined. Let $a = [(m,n)] = [(m',n')]$, $b = [(p,q)] = [(p',q')]$. We have to show that,
		$[(m+p, n+q)] = [(m'+p', n'+q')]$ \hfill(*)\\
		Indeed, as $[(m,n)] = [(m',n')]$, we have $m+n' = n+m'$.\\
		Similarly, $p+q' = q+p'$.\\
		\begin{align*}
			&m+n'+p+q' = n+m'+q+p'\\
			\Rightarrow&(m+p)+(n'+q') = (n+q)+(m'+p')\\
			\Rightarrow&(m+p, n+q) \sim_{\ZZ} (m'+p', n'+q')\\
			\Rightarrow&[(m+p, n+q)] = [(m'+p', n'+q')],\\ 
		\end{align*}
		which proves (*). Hence, $+$ is well-defined.
	\end{proof}

	We now check the associativity of $+$.Let $a,b,c \in \ZZ$ be written as, $a = [(m,n)], b = [(p,q)], c = [(r,s)]$.\\
	Then,\\
	\begin{align*}
		(a+b)+c &= ([(m,n)]+[p,q])+[(r,s)]\\
		&= [(m+p, n+q)]+[(r,s)]\\
		&= [((m+p)+r, (n+q)+s)]\\
		&= [(m+(p+r), n+(q+s))]\\
		&= a+(b+c)
	\end{align*}
 	which shows that $+$ is associative.
 	
 	Now, we show that $+$ is commutative.
 	Let  $a,b \in \ZZ$ be written as, $a = [(m,n)], b = [(p,q)]$.\\
 	Then,\\
 	\begin{align*}
 		a+b &= [(m,n)]+[(p,q)]\\
 		&= [(m+p, n+q)]\\
 		&= [(p+m, q+n)]\\
 		&= b+a
 	\end{align*}
 	which shows that $+$ is commutative.
 	
 	ii) Let $a \in \ZZ$ be written as, $a = [(m,n)]$. Then,\\
 	\begin{align*}
 		a+\overline{0} &= [(m,n)]+[(1,1)]\\
 		&= \underbrace{[(m+1, n+1)] = [(m,n)]}\\
 		&= \overline{a}
 	\end{align*}
 	as, $(m+1, n+1) \sim_{\ZZ} (m,n)$. Also, we have earlier proved the commutativity of $+$ over $\ZZ$. Therefore, we have proved that,
 	$$a+\overline{0} = \overline{a} = \overline{0}+a, \forall\ a \in \ZZ$$
 	
 	iii) Let $a \in \ZZ$ be written as, $a = [(m,n)]$. We define $x \in \ZZ$ as, $x \defeq [(n,m)]$.\\
 	Then,\\
 	\begin{align*}
 		a+x &= [(m,n)]+[(n,m)]\\
 		&= [(m+n, n+m)]\\
 		&= [(m+n, m+n)]\\
 		&= [(1,1)]\\
 		&= \overline{0}
 	\end{align*}
 
 	We now prove the uniqueness of $x$. Let us suppose that $\exists\ x,y \in \ZZ$, such that,\\
 	\begin{center}
 		$a+x = x+a = \overline{0}$, and $a+y = y+a = \overline{0}$\\
 	\end{center}
 	We now show that $x=y$. Indeed, using ii),\\
 	$x = \overline{0} + x = (y+a) +x = y+(a+x) = y+\overline{0} = y $, which proves the uniqueness.\\[9pt]
 	
 	iv) Let $a,b \in \ZZ$ be given.We must define $x = (-a)+b$.\\
 	Then,
 		$$a+x = a+((-a)+b) = (a+(-a))+b = \overline{0}+b = b$$
 		
 	We now prove the  uniqueness of $x$. Let there be $a,b \in \ZZ$. We must define $x,y \in \ZZ$ as $x = (-a)+b$ and $y = (-a)+b$. Then,\\
 	\begin{center}
 		$a+x = a+((-a)+b) = (a+(-a))+b = 0+b = (a+(-a))+b = a+((-a)+b) = a+y$
 	\end{center}
 	This shows that $x=y$, which in turn proves the uniqueness of $x$.\\[10pt]
 	
 	We now establish the properties of multiplication on $\ZZ$.\\[10pt]
 	
 	v) We prove that multiplication on $\ZZ$ is well-defined. Let $a,b \in \ZZ$ be defined as, $a = [(m,n)] = [(m',n')]$ and $b = [(p,q)] = [(p',q')]$.
 	
 	We shall show that,
 	\begin{align*}
 		[(m,n)][(p,q)] &=  [(m',n')][(p',q')]\\
 		\Rightarrow[(mp+nq, mq+np)] &= [(m'p'+n'q', m'q'+n'p')]\\ 
 	\end{align*}
 	$$\Rightarrow \boxed{mp+nq+m'q'+n'p' = m'p'+n'q'+mq+np}\\$$
 	
 	To prove this, we proceed as follows. We have,\\
 	\begin{align}
 		m+n' = m'+n\\
 		p+q' = q+p'
 	\end{align}
 	\begin{flushleft}
 	$(1) \times p$ $\Rightarrow mp+n'p = m'p+np$\\
 	$(1) \times q$ $\Rightarrow mq+n'q = m'q+nq$\\
 	$(2) \times m'$ $\Rightarrow pm'+q'm' = qm'+p'm'$\\
 	$(2) \times n'$ $\Rightarrow pn'+q'n' = qn'+p'n'$\\
 	\end{flushleft}
 	This implies that,
 	\begin{align*}
 		mp+n'p+m'q+nq+pm'+q'm'+qn'+p'n' &= m'p+np+mq+n'q+qm'+p'm'+pn'+q'n'\\
 		(mp+nq+m'q'+n'p')+[n'p+m'q+m'p+n'q] &= (mq+np+m'p'+n'q')+[n'p+m'q+m'p+n'q]
 	\end{align*}
 
 	Now, invoking \texttt{Lemma \ref{lemma:npq}}, we conclude that,
 	$$mp+nq+m'q'+n'p' = mq+np+m'p'+n'q'$$
 	which proves our assumption. Hence, the multiplication is well-defined on $\ZZ$. This proves \texttt{(v)} and \texttt{(vi)}.\\[10pt]
 	
 	(vii) We prove that $\forall a,b,c \in \ZZ$, $a\cdot (b+c) = a\cdot b + a\cdot c$. Let $a,b,c \in \ZZ$ be defined as, $a = [(m,n)]$, $b = [(p,q)]$, and $c = [(r,s)]$.
 	
 	Now, we can say,
 	\begin{align*}
 		a \cdot (b+c) &= [(m,n)]\cdot ([(p,q)] + [(r,s)])\\
 		&= [(m,n)]\cdot[(p+r, s+q)]\\
 		&= [(mp+mr+ns+nq, ms+mq+np+nr)]\\
 		ab+ac &= [(m,n)][(p,q)]+[(m,n)][(r,s)]\\
 		&= [(mp+nq, mq+np)]+[(mr+ns, ms+nr)]\\
 		&= [(mp+nq+mr+ns, mq+np+ms+nr)]
 	\end{align*}
 
	Since, $a \cdot (b+c) = [(mp+mr+ns+nq, ms+mq+np+nr)] = [(mp+nq+mr+ns, mq+np+ms+nr)] = ab+ac$, we have proved the claim that,
	$$\forall\ a,b,c \in \ZZ,\ a\cdot (b+c) = a\cdot b + a\cdot c$$
	
	\begin{center}
		\texttt{This completes the proof for Theorem \ref{theorem:ring}.}
	\end{center}

	Before we proceed further, let's introduce the following notation.\\[10pt]
	\textbf{\underline{Notation: }}\\
	We write,
	$$\boxed{\ZZ^{+} \defeq \{[(j,1)]: j \in \NN, j\geq 2\}}$$
	
	\begin{theorem}
		\texttt{\textbf{Embedding of $\NN$: }}\\
		Define $f:\NN \mapsto \ZZ$ by,
		$$\boxed{f(n) \defeq [(n+1,1)] \forall\ n \in \NN}$$
		Then $f$ satisfies the following properties:
		\begin{enumerate}[i)]
			\item $f$ is one-one(injective),
			\item $f(\NN) = \ZZ^{+}$,
			\item $f(1) = \overline{1}$,
			\item $\forall\ m,n \in \NN$, $$f(m+n)= f(m)+f(n),\ f(mn) = f(m)\cdot f(n)$$
		\end{enumerate}
	\label{theorem:embed}
	\end{theorem}
	\begin{proof}
		\begin{enumerate}[i)]
			\item Let there be $x,y \in \NN$ such that $f(x)=f(y)$.
			\begin{align*}
				[(x+1,1)] &= [(y+1,1)]\\
				\Rightarrow (x+1)+1 &= 1+(y+1)\\
				\Rightarrow x+1+1 &= y+1+1\\
				\Rightarrow x &= y
			\end{align*}
			This proves that $f(n)$ is indeed one-one or, injective.
			
			\item We now show that, $f(\NN) = \ZZ^{+}$. By definition, $f(\NN) = \{f(n):n \in \NN\} = [(n+1,1)] \forall\ n \in \NN$. Now, $\ZZ^{+}$ is defined as, $\ZZ^{+} = \{[(j,1)]: j \in \NN, j\geq 2\}$.
			
			We know from the \texttt{Well-Ordering Property} of $\NN$, that the minimum element of $\NN$ is $1$. Thus, we can say that $n+1,\ \forall\ n\in \NN$ is greater than $2$. This makes the fact evident that,
			$$f(\NN) = \{f(n):n \in \NN\} = \{[(n+1,1)]: n \in \NN,\ n+1 \geq 2\} = \{[(j,1)]: j \in \NN, j\geq 2\} = \ZZ^{+} $$
			$$\Rightarrow f(\NN) = \ZZ^{+}$$
			This proves the claim.
			
			\item We now show that $f(1) = \overline{1}$.
			\begin{align*}
				f(n) &= [(n+1,1)] \forall\ n \in \NN\\
				\Rightarrow f(1) &= [(2,1)]\\
				\overline{1} &= [(2,1)],\ from\ \texttt{Definition}\ \ref{definition:bar}\\
				\Rightarrow f(1) &= \overline{1}
			\end{align*}
		This proves our claim.
		
		\item We need to show that, $\forall\ m,n \in \NN$, $f(m+n)= f(m)+f(n),\ f(mn) = f(m)\cdot f(n)$.\\
		Let's first try proving,  $f(m+n)= f(m)+f(n)$.
		\begin{align*}
			f(m+n) &= [(m+n+1,1)]\\
			f(m) &= [(m+1,1)]\\
			f(n) &= [(n+1, 1)]\\
			\Rightarrow f(m)+f(n) &= [(m+1,1)] + [(n+1, 1)]\\
			&= [(m+n+1+1,1+1)]\\
			&= [(m+n+1,1)]\\
			\Rightarrow f(m+n) &= f(m) + f(n)
		\end{align*}
	This proves our claim.
		\end{enumerate}
	\end{proof}

	\begin{corollary}
		Let $f: \NN \mapsto \ZZ$ be the map defined in Theorem \ref{theorem:embed}. Then,
		$$\ZZ = \{f(n):n \in \NN\} \cup \{-f(n):n \in \NN\} \cup \overline{0}$$
	\end{corollary}
	\textbf{\underline{Convention: }}\\
 	Let $f: \NN \mapsto \ZZ$ be the embedding map defined in Theorem \ref{theorem:embed}. We shall identify $f(n)$ with $n,\ \forall\ n \in \NN$. Then, $\ZZ = \{n \in \NN\} \cup \{-n|n \in \NN\} \cup \{\overline{0}\}$
 	
 	\begin{theorem}
 		\texttt{\textbf{Order in $\ZZ$}}\\
 		For all $a,b \in \ZZ$, we can say that,
 		{\color{Red}
 		\begin{enumerate}
 			\item $a>b$ iff $\exists\ x\in \ZZ^{+}$ such that $b+x = a$
 			\item $a \geq b$ iff either $a=b$ or $a>b$
 		\end{enumerate}
 		}
 	\end{theorem}
 
 \section{Rationals}
 
 We conclude the chapter by constructing the set of Rational Numbers out of the set of Integers. The construction, in this case as well, proceeds with an appropriate equivalence relation.
 
 \begin{definition}
 	$\QQ$ - Equivalence Relation\\
 	Define $\sim_{\QQ}$ on $\ZZ \times \left(\ZZ \setminus \{0\}\right)$ by, $\forall (a,b),(p,q) \in (\ZZ \setminus \{ 0 \} )$
 	$$(a,b) \sim_{\QQ} (p,q) \Leftrightarrow aq = bp$$.
 \end{definition}

\begin{lemma}
	$\sim_{\QQ}$ is an equivalence relation on $\ZZ \times (\ZZ \setminus \{0\})$.
\end{lemma}

\begin{proof}
	We now check for the three conditions, viz. Reflexivity, Symmetry and Transitivity for the relation $\sim_{\QQ} \forall (a,b),(p,q) \in \ZZ \times (\ZZ \setminus \{0\})$.
	\begin{enumerate}[(i)]
		\item Reflexivity:\\
		 We know that $(a,b) \sim_{\QQ} (p,q)$ iff $aq=bp$.\\
		 This implies that $(a,b)\sim_{\QQ}(a,b) \Leftrightarrow ab=ba$, which is true from Theorem \ref{theorem:ring}.\\
		 Therefore, it is proven that $\sim_{\QQ}$ is Reflexive.
		 
		 \item Symmetry:\\
		 We know now that,
		 \begin{align*}
		 	(a,b) \sim_{\QQ} (p,q) \Leftrightarrow aq=bp \\
		 	(p,q) \sim_{\QQ} (a,b) \Leftrightarrow pb=qa
		 \end{align*}
	 	Since both the above relations lead to the same result, we can conclude that $\sim_{\QQ}$ is Symmetric.
	 	
	 	\item Transitivity:\\
	 	We know now that,
	 	\begin{align*}
	 		(a,b) \sim_{\QQ} (p,q) &\Leftrightarrow aq=bp \\
	 		(p,q) \sim_{\QQ} (r,s) &\Leftrightarrow ps=qr \\
	 		\Rightarrow (a,b) \sim_{\QQ} (r,s) &\Leftrightarrow as=br
	 	\end{align*}
	\end{enumerate}
\end{proof}
\end{document}

