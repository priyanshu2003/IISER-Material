\documentclass[11pt]{scrartcl}
\usepackage{answers}
\Newassociation{sketch}{hintitem}{hints}
\renewcommand{\solutionextension}{out}
\usepackage[none]{hyphenat}
%\usepackage[margin=1in]{geometry}

\usepackage[sexy]{evan}
\newcommand\EE{\mathbb E}
\newcommand\PP{\mathbb P}
%\setlength{\parindent}{0em}
%\setlength{\parskip}{2.5em}

\begin{document}
	\title{Module 2: Number Systems}
	\author{Priyanshu Mahato}
	\date{\today}
	\maketitle
	
	\begin{abstract}
		Email: \mailto{pm21ms002@iiserkol.ac.in}.These are my personal notes on Number Systems. We will consider $\mathbb{N}$ (Natural Numbers), $\mathbb{Z}$ (Integers), and $\mathbb{Q}$ (rational Numbers), but not $\mathbb{R}$ (Real Numbers).
	\end{abstract}

	\section{Natural Numbers}
	
		The natural numbers are $1,2,3,4,\dots$. The set of all natural numbers is denoted by $\NN$.
		
		\begin{definition}
			We assume familiarity with the algebraic operations of addition and multiplication on the set $\NN$ and also with the linear order relation $<$ on $\NN$ defined by ``$a<b$ if $a, b \in \NN$ and $a$ is less than $b$".
		\end{definition}
	
		We discuss the following fundamental properties of the set $\NN$.
		
		\begin{enumerate}
			\item Well Ordering Property
			\item  Principle of Induction
		\end{enumerate}
	
	\subsection{Well Ordering Property}
		\begin{definition}
			Every non-empty subset of $\NN$ has a least element.
		\end{definition}
		This means that if $S$ is a non-empty subset of $\NN$, then there is an element $m$ in $S$ such that $m \leq s$ for all $s \in S$.
		
		In particular, $\NN$ itself has the least element 1.
		
		\begin{proof}
			Let $S$ be a non-empty subset of $\NN$. Let $k$ be an element of $S$. Then $k$ is a natural number.
			
			We define a subset $T$ by $T = \left\{ x \in S : x \leq k \right\}$. The $T$ is a non-empty subset of $\left\{ 1,2,3,\dots, k \right\}$.
			
			Clearly, $T$ is a finite subset of $\NN$ and therefore it has a least element, say $m$. Then $1 \leq m \leq k$.
			
			We now show that $m$ is the least element of $S$. Let $s$ be any element of $S$.
			
			If $s>k$, then the inequality $m \leq k$ implies $m<s$.
			
			If $s \leq k$, the $s \in T$; and $m$ being the least element of $T$, we have $m \leq s$.
			
			Thus $m$ is the least element of $S$.
		\end{proof}
	
	\subsection{Principle of Induction}
	\begin{definition}
		Let $S$ be a subset of $\NN$ such that,
		\begin{enumerate}[i)]
			\item  $1 \in S$ and,
			\item  if $k \in S$, then $k+1 \in S$.
		\end{enumerate}
	Then $S = \NN$
	\end{definition}
	
	\begin{proof}
		Let $T = \NN - S$. We prove that $T = \phi$.
		
		Let $T$ be non-empty. then by the \emph{Well Ordering Property} of $\NN$, the non-empty subset $T$ has a least element, say $m$.
		
		Since $1 \in S$ and $1$ is the least element of $\NN$, $m>1$.
		
		Hence, $m-1$ is a natural number and $m-1 \notin T$. So, $m-1 \in S$.
		
		But by ii) $m-1 \in S \Rightarrow (m-1)+1 \in S$, i.e., $m \in S$.
		
		This contradicts that $m$ is the least element in $T$. Therefore, our assumption is wrong and $T = \phi$.
		
		Therefore, $S=\NN$.
	\end{proof}
		 
	\begin{theorem}
		Let $P(n)$ be a statement involving a natural number $n$. If,
		\begin{enumerate}[i)]
			\item $P(1)$ is true, and
			\item $P(k+1)$ is true whenever $P(k)$ is true,
		\end{enumerate}
		then $P(n)$ is true for all $n \in \NN$.
	\end{theorem}

	\begin{proof}
		Let $S$ be the set of those natural numbers for which the statement $P(n)$ is true.
		
		Then $S$ has the properties,
		\begin{enumerate}[(a)]
			\item $1 \in S$, by (i)
			\item $k \in S \Rightarrow k+1 \in S$ by (ii).
		\end{enumerate}
	
		By the \emph{Principle of Induction}, $S = \NN$.
		
		Therefore, $P(n)$ is true for all $n \in \NN$.
	\end{proof}

	\begin{remark}
		Let a statement $P(n)$ satisfies the conditions,
		\begin{enumerate}[(i)]
			\item for some $m \in \NN,\ P(m)$ is true ($m$ being the least possible); and
			\item $P(k)$ is true $\Rightarrow$ $P(k+1)$ is true for all $k \geq m$.
		\end{enumerate}
		Then $P(n)$ is true for all natural numbers $\geq m$.
	\end{remark}

	\pagebreak

	\begin{flushleft}
			\textbf{Worked Examples}
	\end{flushleft}
	\begin{example}
		Prove that for each $n \in \NN$, $1+2+3+4+\dots +n = \frac{n(n+1)}{2}$ for all $n \in \NN$.
	\end{example}

	The statement is true for $n=1$, because $1 = \dfrac{1(1+1)}{2}$.\\
	Let the statement be true for some natural number $k$.\\
	Then $1+2+3+4+\dots +k = \dfrac{(k+1)}{2}$ and therefore,\\
	$(1+2+\dots+k)+(k+1) = \dfrac{k(k+1)}{2} + (k+1)$\\
	or, $1 +2 +3 + \dots +(k+1) = \dfrac{(k+1)(k+2)}{2}$.
	
	This shows that the statement is true for the natural number $k+1$ if it is true for k. By the principle of induction, the statement is true for all natural numbers.
	
	\begin{example}
		Prove that for each $n \geq 2,\ (n+1)! > 2^{n}$.
	\end{example}

	The equality holds for $n=2$ since $(2+1)! > 2^{2}$.
	
	Let the inequality hold for some natural number $k \geq 2$.
	
	Then, $(k+1)! > 2^{k}$
	\begin{align*}
		and\ (k+2)! &= (k+2)(k+1)!\\
		&> 2 \cdot 2^{k}, since\ k+2>2\\
		or,\ (k+2)! &> 2^{k+1}
	\end{align*}

	This shows that if the inequality holds for $k(\geq 2)$ then it also holds for $k+1$.
	
	By the principle of induction, the inequality holds for all natural numbers $\geq 2$.
	
	[Note that the inequality does not hold for $n=1$.]
	
	\subsection{Second Principle of Induction (or, Principle of Strong Induction)}
	\begin{definition}
		Let $S$ be a subset of $\NN$ such that
		\begin{enumerate}[(i)]
			\item $1 \in S$, and
			\item if $\{1,2,3,4,\dots\} \subset S$, then $k+1 \in S$.
		\end{enumerate}
		Then $S = \NN$
	\end{definition}

	\begin{proof}
		Let $T = \NN - S$. We prove that $T = \phi$.
		
		Let $T$ be non-empty. Then $T$ will have a least element, say $m$, by the WOP of $\NN$. Since, $1 \in S,\ 1 \notin T$.
		
		As $m$ is the least element in $T$ and $1 \notin T$, $m>1$.
		
		By choice of $m$, all natural numbers less than $m$ belong to $S$. That is $1,2,\dots, m-1$ all belong to $S$.
		
		Then by (ii) $m \in S$ and consequently, $m \notin T$, a contradiction. It follows that $T = \phi$ and therefore, $S=\NN$.
	\end{proof}

\pagebreak

	\begin{flushleft}
			\textbf{Worked Examples}(continued)
	\end{flushleft}
	
	\begin{example}
		Prove that for all $n \in \NN$, $(3+\sqrt{5})^{n} + (3-\sqrt{5})^{n}$ is an even integer.
	\end{example}
	Let $P(n)$ be the statement ``$(3+\sqrt{5})^{n} + (3-\sqrt{5})^{n}$ is an even integer".
	
	$P(1)$ is true since $(3+\sqrt{5})^{1} + (3-\sqrt{5})^{1} = 6$, an even integer.
	
	Let us assume that $P(n)$ is true for $n=1,2,\dots,k$.
		\begin{align*}
			&(3+\sqrt{5})^{(k+1)} + (3-\sqrt{5})^{(k+1)}\\
			&= a^{(k+1)} + b^{(k+1)}\ where\ a=3+\sqrt{5},\ b=3-\sqrt{5}\\
			&= (a^{k} + b^{k})(a+b) - (a^{k-1} + b^{k-1})ab\\
			&= 6(a^{k} + b^{k}) - 4(a^{k-1} + b^{k-1}).		
		\end{align*}
	
	It is an even integer, since $a^{k}+b^{k}$ and $a^{k-1}+b^{k-1}$ are even integers.
	
	Hence, $P(k+1)$ is true whenever $P(n)$ is true for all $n=1,2,\dots,k$.
	
	By the second principle of induction, $P(n)$ is true for all natural numbers.
	
	\section{Integers}
	
	We shall now construct the set of integers using the set of Natural Numbers. Our construction will be through an equivalence relation on $\NN \times \NN$.
	
	\begin{definition}
		Define $\sim_{\ZZ}$ on $\NN \times \NN$ by, for all $(m,n),(p,q) \in \NN \times \NN$,
		 $$(m,n)\sim_{\ZZ}(p,q) \Leftrightarrow m+q = n+p$$
	\end{definition}

	\begin{lemma}
		\begin{enumerate}[i)]
			\item $\sim_{\ZZ}$ is an equivalence relation on $\NN \times \NN$.
			\item for all $(m,n) \in \NN \times \NN$,
				\[(m,n) \sim_{\ZZ}
				\begin{cases*}
					$(m+1-n, 1)$ & for $m\geq n$\\
					$(1, n+1-m)$ & for $n\geq m$
				\end{cases*}\]
			\item $\NN \times \NN / \sim_{\ZZ} = \{[(j,1)] : j \in \NN\ \&\ j\geq 2\} \cup \{[(1,k)] : k \in \NN\ \&\ k\geq 2\} \cup \{[(1,1)]\}$
		\end{enumerate}
	\end{lemma}

	\begin{proof}
		Look at MA1101 ps2 (\textbf{Problem 2})
	\end{proof}

	\begin{definition}
		Let us write $\ZZ \defeq \NN \times \NN / \sim_{\ZZ} = \{[(m,n)] : (m,n) \in \NN \times \NN\}$
		
		We also write,
		$$\boxed{\overline{0} \defeq [(1,1)]\ \&\ \overline{1} \defeq [(2,1)]}$$
		
		Let $\overline{a} \defeq [(m,n)]$, $\overline{b} \defeq [(p,q)] \in \ZZ$.
		\begin{enumerate}[i)]
			\item \textbf{Addition}\\
			$$\boxed{ a + b \defeq [(m+p, n+q)]}$$
			\item \textbf{Multiplication}\\
			$$\boxed{ a \cdot b \defeq [(mp+nq, mq+np)]}$$
		\end{enumerate}
	\end{definition}

	We have the following important theorem,
	
	\begin{theorem}
		\begin{enumerate}[i)]
			\item $+$ is well-defined, commutative and associative
			\item $a+\overline{0} = a = \overline{0} + a$, $\forall a \in \ZZ$
			\item $\forall a \in \ZZ, \exists$ a unique $x \in \ZZ$, such that $a + x = \overline{0}$. We write $-a$ for $x$ and say that $-a$  is  the negative of $a$
			\item $\forall a,b \in \ZZ, \exists$ a unique $x \in \ZZ$ such that $a+x = b$
			\item $\cdot$ is well defined, associative, and commutative
			\item $a\cdot\overline{1} = a = \overline{1}\cdot a$, $\forall a \in \ZZ$
			\item $\forall a,b \in \ZZ$, $a\cdot (b+c) = a\cdot b + a\cdot c$
		\end{enumerate}
	\label{theorem:ring}
	\end{theorem}

	\begin{remark}
		In other words, we can call $(\ZZ, +, \cdot)$ as a commutative \textbf{ring} with identity.
	\end{remark}

	To prove \ref{theorem:ring}, we start off with a lemma,
	
	\begin{lemma}
		$\forall\ n,p,q \in \NN$, if $n+p = n+q$ $\Rightarrow p=q$
	\end{lemma}
\end{document}

