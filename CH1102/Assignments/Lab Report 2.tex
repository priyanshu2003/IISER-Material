\documentclass[12pt]{article}

\usepackage[margin=1in]{geometry}
\usepackage{amsfonts, amssymb, amsmath}
\usepackage[none]{hyphenat}
\usepackage{fancyhdr}
\usepackage{float}
\usepackage{setspace}
\usepackage{graphicx}

\pagestyle{fancy}
\fancyhead{}
\fancyfoot{}
\fancyhead[L]{\MakeUppercase{CH1102 Assignment-1}}
\fancyhead[R]{\slshape Priyanshu Mahato}
\fancyfoot[C]{\thepage}

\renewcommand{\footrulewidth}{1pt}
% \renewcommand{\baselinestretch}{1.2}

\setlength{\headheight}{14.5pt}
\setlength{\parindent}{0em}
\setlength{\parskip}{3em}

\begin{document}
	\thispagestyle{empty}
	\begin{titlepage}
		\begin{center}
			\vspace{2cm}
			\huge\textbf{CH1102}\\
			\vspace{1cm}
			\large\textbf{Chemistry Laboratory I}
			\vfill
			\line(1, 0){470}\\[14pt]
			\huge\textbf{Lab Report - 2}\\[10pt]
			\Large\textbf{Expt. No. 2: Acid Neutralisation Power of Common Antacids}\\[14pt]
			\line(1, 0){470}
			\vfill
			By: Priyanshu Mahato\\
			Roll No.: pm21ms002\\
			\today
			
		\end{center}
	\end{titlepage}
	
	\setcounter{page}{1}
	
	\large\textbf{\centering Experiment No. 02: Acid Neutralisation Power of Common Antacids}
	
	\textbf{Aim: }Determination of acid neutralisation power of common (commercially available) antacids.
	
	\textbf{Objectives: }\\
	• Understand standardization of acids and bases by titration and perform titration calculations.\\
	• Reinforce the procedure of acid/base titration (using phenolphthalein)\\
	• Introduce the concept of buffers\\
	• Introduce the concept of back-titration\\
	• Determine the acid neutralizing power of two commercial antacids.\\
	• Compare theoretical and experimental results.\\
	
	\textbf{Apparatus Required: }\\
	$\bullet$ Burette\\
	$\bullet$ Pipette\\
	$\bullet$ Beaker\\
	$\bullet$ Conical Flask\\
	
	\textbf{Chemicals Required: }\\
	$\bullet$ Common Antacid(Diazen, Rantac, Milk of Magnesia, Gelusil)\\
	$\bullet$ Potassium Hydrogen Phthalate(KHP)\\
	$\bullet$ Standardized HCl solution\\
	$\bullet$ Standardized NaOH solution\\
	$\bullet$ Bromothymol Blue(BTB)\\
	
	\pagebreak
	
	{\Large\textbf{Experimental Procedure: }}
	
	$\bullet$ Put some water in the burette and practice controlling the stopcock. Do not fill burettes on the work-bench. Always keep all chemicals below eye level. This decreases the chance of getting chemicals in your eye in the event of a spill.\\
	$\bullet$ If you have air bubbles in the burette, gently knock the bottom of the burette to free them so they can rise to the 
	surface.\\
	$\bullet$ You will determine the volume of titrant delivered by subtracting the initial buret reading from the final (volume by difference).\\
	$\bullet$ Mount the burette on the stand. In real titrations, you would put a white towel or piece of paper over the dark base of the ring stand so the color change of the indicator will be easy to see. Since this is a practice, your titrant is water. You’re just practicing the stopcock control and volume reading. The goal is to get a feel for the burette.\\
	$\bullet$ Practice reading the volume (liquid level at the bottom of the meniscus). Take readings to 0.01 or 0.02 ml.\\
	$\bullet$ Record the initial volume of water. Add water to a collection flask and read the new volume. Find the volume of 
	water added by difference.\\
	$\bullet$ Practice by delivering a milliliter, a few drops, and one drop.\\
	$\bullet$ Set up a 50-ml burette with the stock NaOH. It may help you to start with Part 3 because it takes some time for the solution to heat up and cool.\\
	
	\textbf{Part 1: Standardization of NaOH}\\[14pt]
	Determine the concentration of the base, NaOH, by titrating a known mass of KHP, to neutral (the equivalence point). 
	The molar mass of KHP is 204.23 g/mol, and it has one acidic hydrogen per molecule.
	
	Prepare 100 ml 0.01 (N) KHP. About 10 ml of NaOH should be used in the titrations. Use a few drops of BTB as indicator in the titration flask. Record the initial volume of NaOH from the burette and then begin the titration. As you 
	turn the stopcock, push it into the barrel so it doesn’t loosen and leak. Record the color change at the end point and 
	the final volume on the burette. The volume of NaOH used $= V_{final}–V_{initial}$. Perform three titrations with the NaOH to 
	obtain reproducible results.
	
	\textbf{Part 2: Standardization of HCl}\\[14pt]
	To determine the precise molarity of the HCl solution, titrate it with the NaOH to the endpoint; use BTB as the indicator 
	unless instructed otherwise. Use a volumetric pipette to transfer exactly 10 ml of stock HCl into a 125 ml Erlenmeyer 
	flask. Record the initial volume of NaOH and titrate the HCl.
	
	Record the color change at the end point and the final volume of NaOH. The volume of NaOH used $= V_{final}–V_{initial}.$
	Repeat to be sure you can get reproducible results.
	
	\textbf{Part 3: Determination of the Amount of Acid Neutralized by an Antacid Tablet}\\[14pt]
	Take 25 ml of supplied antacid solution in conical flask. Using a volumetric pipette, accurately add 25 ml of standardized HCl solution. Heat the solution gently to a near boil for about 5 minutes and carefully avoid splattering. Allow the solutions to cool (to touch). Add a few drops of BTB indicator. Titrate the solution with standardized NaOH to reach the endpoint.
	
	Calculate the number of moles of HCl, $nH^{+}$, added to the solution. Next, calculate the number of moles of NaOH 
	titrant that required for titration. This is the number of moles of HCl neutralized by the NaOH. Determine the number of moles of HCl not neutralized by the NaOH to find the number of moles of HCl neutralized 
	by the antacid.
	
	\pagebreak
	
	\textbf{Results and Observations: }\\[16pt]
	Preparation of 100 ml 0.1 (N) KHP solution.\\[14pt]
	\textbf{Weight Taken: }2.04g\\
	\textbf{Concentration of KHP solution: }0.1M\\
	\textbf{Preparation of 0.1N NaOH solution: }0.1N supplied\\
	\textbf{Preparation of 0.1N HCl solution: }0.1N supplied\\
	
		\begin{table}[H]
		\centering
		\def\arraystretch{1.5}
		\begin{tabular}{|c||c|c|}
			\hline
			$No.$&Volume of KHP&Volume of NaOH\\
			\hline
			\hline
			$1$&$25\ mL$&$26.2\ mL$\\
			\hline
			$2$&$25\ mL$&$26.4\ mL$\\
			\hline
			$3$&$25\ mL$&$26.2\ mL$\\
			\hline
		\end{tabular}
		\caption{Standardization of NaOH with standard KHP}
		\label{tab:std_NaOH}
	\end{table}

	\textbf{Concentration of NaOH: }
	
	$$V_{KHP}\ \cdot S_{KHP} = V_{NaOH} \cdot S_{NaOH}$$
	$$\Rightarrow 25mL \cdot 0.1N = \frac{26.2+26.4+26.2}{3}\ mL \cdot S_{NaOH}$$
	$$\Rightarrow S_{NaOH} = \underline{0.095N}$$
	
		\begin{table}[H]
		\centering
		\def\arraystretch{1.5}
		\begin{tabular}{|c||c|c|}
			\hline
			$No.$&Volume of NaOH&Volume of HCl\\
			\hline
			\hline
			$1$&$25.2\ mL$&$25\ mL$\\
			\hline
			$2$&$25.1\ mL$&$25\ mL$\\
			\hline
			$3$&$25.1\ mL$&$25\ mL$\\
			\hline
		\end{tabular}
		\caption{Standardization of HCl with previously standardized NaOH solution}
		\label{tab:std_HCl}
	\end{table}

	\pagebreak

	\textbf{Concentration of HCl: }
	
	$$V_{HCl} \cdot S_{HCl} = V_{NaOH} \cdot S_{NaOH}$$
	$$\Rightarrow 25mL \cdot S_{HCl} = \frac{25.2+25.1+25.1}{3} mL \cdot 0.095N$$
	$$\Rightarrow S_{HCl} = 0.0955 \approx \underline{0.096}$$
	
	
		\begin{table}[H]
		\centering
		\def\arraystretch{1.5}
		\begin{tabular}{|c||c|c|c|}
			\hline
			$No.$&Volume of Antacid&Volume of HCl&Volume of NaOH\\
			\hline
			\hline
			$1$&$25\ mL$&$25\ mL$&$11.2\ mL$\\
			\hline
			$2$&$25\ mL$&$25\ mL$&$11.1\ mL$\\
			\hline
			$3$&$25\ mL$&$25\ mL$&$11.2\ mL$\\
			\hline
		\end{tabular}
		\caption{Calculation of Antacid concentration by back titration}
		\label{tab:back}
	\end{table}
	
	\textbf{Amount of acid neutralizing by the antacid: }
	
	$$n_{HCl} (neutralized\ by\ the\ antacid) = (V_{HCl} \cdot S_{HCl}) - (V_{NaOH} \cdot S_{NaOH})$$
	$$\Rightarrow n_{HCl} = (25mL \cdot 0.096N)-\left(\frac{11.2+11.1+11.2}{3}mL \cdot 0.095N\right)$$
	$$\Rightarrow n_{HCl} = (25mL \cdot 0.096N)-\left(11.17mL \cdot 0.095N\right)$$
	$$\Rightarrow n_{HCl}(neutralized\ by\ the\ antacid) = n_{antacid} =  1.33885\ mmol$$
	$$\Rightarrow S_{antacid} = \frac{n_{antacid}}{V_{antacid}}$$
	$$\Rightarrow S_{antacid} = \frac{1.33885}{25}\ N$$
	$$\Rightarrow S_{antacid} = 0.053554N \approx 0.05N$$
	
	\pagebreak
	
	{\Large\textbf{Conclusions: }}
	
	1. \textbf{Possible experimental reasons for error (deviations from expected values):-} \\
	$\bullet$ \textbf{End Point Error}: The end point of a titration is when the reaction between the two solutions has stopped. Indicators, which change color to indicate when the reaction has stopped, do not change instantly. In the case of acid-base titration, the indicator may first lighten in color before changing completely. Also, each individual perceives color slightly differently, which affects the outcome of the experiment. If the color has changed slightly, too much of the titrant, which comes from the burette, can be introduced into the solution, overshooting results.\\
	$\bullet$ \textbf{Misreading the Volume}: The accuracy of titration requires precise measurement of the volume of materials in use. But markings on a burette can be easily misread. One way to misread the volume is by looking at the measurement on an angle. From above, it can seem like the volume is lower, while from below, the apparent volume looks higher. Another source of measurement error is looking at the wrong spot. A solution forms a concave curve and the bottom of the curve is used to measure the volume. If the reading is taken from the higher sections of the curve, the volume measurement will be in error.\\
	$\bullet$ \textbf{Concentrations}: Errors in concentrations directly affect the measurement accuracy. Errors include using the wrong concentration to begin with, which can occur from chemical decomposition or evaporation of fluids. The solution may have been prepared incorrectly or contaminatns could have been introduced into the solution, such as using dirty equipment. Even the process of cleaning your equipment, if carried out with the wrong solution, can affect the concentrations of the solutions to be experimented on.\\
	$\bullet$ \textbf{Using the Equipment Incorrectly}: You must follow strict guidelines in handling and using all equipment during the experiment as the slightest mistake can create errors in the findings. For example, swirling the solution can result in loss of solution that will affect results. Errors in filling the burette can cause air bubbles that affect the flow of the liquid in the burette.\\
	$\bullet$ \textbf{Other Errors}: Other human or equipment errors can also creep in. Human error includes using selecting the wrong reagents or using the wrong amount of indicator. Equipment error typically is in the burette, which can develop leaks over time. Even a small loss of fluid will affect the results of the titration.\\
	$\bullet$ \textbf{Human Errors}: A few errors in chemistry experiments are due simply to mistakes on the part of the person performing the work. There are an endless number of potential mistakes in lab work, but some of the most common include misreading gauges, making math mistakes during dilutions and other types of calculations and spilling chemicals during transfer. Depending on the type of mistake and the stage at which it happens, the associated degree of error in the experimental results will vary widely in magnitude.\\
	$\bullet$ \textbf{Improper Calibrations}: Incorrect or non-existent calibration of instruments is another common source of error in chemistry. Calibration is the process of adjusting or checking an instrument to ensure that the readings it gives are accurate. To calibrate a weigh scale, for example, you might place an object known to weigh 10 grams on the scale, then check that the scale reads 10 grams. Instruments which are not calibrated or are improperly calibrated are not uncommon in chemical labs and lead to wrong results.\\
	$\bullet$ \textbf{Measurement Estimation}: In the expanded meaning of "error" in science, the process of estimating a measurement is considered a source of error. For example, a technician filling a beaker with water to a given volume has to watch the water level and stop when it is level with the filling line marked on the container. Unavoidably, even the most careful technician will sometimes be slightly over or below the mark even if only by a very small amount. Similar errors also occur in other circumstances, such as when estimating the end point of a reaction by looking for a specific color change in the reacting chemicals.\\
	$\bullet$ \textbf{Measurement Device Limitations}: Chemists also consider the limitations of measurement equipment in a lab as a source of error. Every instrument or device, no matter how accurate, will have some degree of imprecision associated with it. For example, a measuring flask is provided by the manufacturer with an acknowledged imprecision of from 1 to 5 percent. Using this glassware to make measurements in a lab therefore introduces an error based on that imprecision. In the same manner, other instruments such as weigh scales also have inherent imprecision that unavoidably causes some error.\\
	
	2. We standardized HCl and NaOH solutions using the usual titration methods, and now, we have used the process of \textbf{Back Titration} to measure the concentration of the antacid.
	
	
\end{document}