\documentclass[12pt]{article}

\usepackage[margin=1in]{geometry}
\usepackage{amsfonts, amssymb, amsmath}
\usepackage[none]{hyphenat}
\usepackage{fancyhdr}
\usepackage{float}
\usepackage{setspace}
\usepackage{graphicx}

\pagestyle{fancy}
\fancyhead{}
\fancyfoot{}
\fancyhead[L]{\MakeUppercase{CH1102 Assignment-1}}
\fancyhead[R]{\slshape Priyanshu Mahato}
\fancyfoot[C]{\thepage}

\renewcommand{\footrulewidth}{1pt}
% \renewcommand{\baselinestretch}{1.2}

\setlength{\headheight}{14.5pt}
\setlength{\parindent}{0em}
\setlength{\parskip}{3em}

\begin{document}
	\thispagestyle{empty}
	\begin{titlepage}
		\begin{center}
			\vspace{2cm}
			\huge\textbf{CH1102}\\
			\vspace{1cm}
			\large\textbf{Chemistry Laboratory I}
			\vfill
			\line(1, 0){470}\\[14pt]
			\huge\textbf{Assignment - 2}\\[10pt]
			\Large\textbf{Expt. No. 2: Acid Neutralisation Power of Common Antacids}\\[14pt]
			\line(1, 0){470}
			\vfill
			By: Priyanshu Mahato\\
			Roll No.: pm21ms002\\
			\today
			
		\end{center}
	\end{titlepage}
	
	\setcounter{page}{1}
	
	\large\textbf{\centering Experiment No. 02: Acid Neutralisation Power of Common Antacids}
	
	\textbf{Question 1.} Define acidimetry and alkalimetry.
	
	\textbf{Answer: } Acidimetry, essentially involves the direct or residual titrimetric analysis of alkaline substances (bases) employing an aliquot of acid and is provided usually in the analytical control of a large number of substances included in the various official compendia. Acidic substances are usually determined quantitatively by methods similar to those used for the quantitative determinations of bases.
	
	The term alkalimetry refers to that part of volumetric chemical analysis which enables us to work out the concentration of an acid solution using an alkaline solution at a known concentration and a suitable indicator.
	
	\textbf{Question 2.} Define acid-base indicator. Name few indicators used at $pH<7\ and\ pH>7$.
	
	\textbf{Answer: }An acid-base indicator is either a weak acid or weak base that exhibits a color change as the concentration of hydrogen ($H^{+}$) or hydroxide ($OH^{-}$) ions changes in an aqueous solution. Acid-base indicators are most often used in a titration to identify the endpoint of an acid-base reaction. They are also used to gauge pH values.
	
	\textbf{$pH<7$: }Thymol Blue(first change), Methyl Orange, Bromocresol Green, etc.
	
	\textbf{$pH>7$: }Phenolphthalein, Thymol Blue(second change), etc.
	
	\pagebreak
	
	\textbf{Question 3.} 25 mL of caustic soda (0.1 N) neutralizes 20 mL of an acid containing 7.875 g acid per litre. Calculate the equivalent mass of the acid.
	
	\textbf{Answer: }\\
	$$V_{NaOH} \cdot S_{NaOH} = V_{acid} \cdot S_{acid}$$
	$$\Rightarrow 25mL \cdot 0.1N = 20mL \cdot 7.875gL^{-1}$$
	$$\Rightarrow \frac{25}{1000} \cdot 0.1 = \frac{20}{1000} \cdot \frac{7.875}{Eq.\ Mass} \cdot 1$$
	$$\Rightarrow Eq.\ Mass = 63$$
	
	\textbf{Question 4.} Calculate the normality of a solution obtained by mixing 100 mL 0.1 N $H_{2}SO_{4}$, 0.5 N $HNO_{3}$ and 0.2 N $HCl$ solutions.
	
	\textbf{Answer: }
	$$N_{R} = \frac{[N_{a}V_{a} + N_{b}V_{b} + N_{c}V{c}]}{[V_{a}+V_{b}+V_{c}]}$$
	Here,
	$$N_{sol^{\underline{n}}} = \frac{[N_{H_2SO_4}V_{H_2SO_4} + N_{HNO_3}V_{HNO_3} + N_{HCl}V_{HCl}]}{[V_{H_2SO_4}+V_{HNO_3}+V_{HCl}]}$$
	$$\Rightarrow N_{sol^{\underline{n}}} = \frac{[0.1N \cdot 100mL + 0.5N \cdot 100mL + 0.2N \cdot 100mL]}{[100mL+100mL+100mL]}$$
	$$\Rightarrow N_{sol^{\underline{n}}} = \frac{[0.1N \cdot 100mL + 0.5N \cdot 100mL + 0.2N \cdot 100mL]}{300mL}$$
	$$\Rightarrow N_{sol^{\underline{n}}} = \frac{[10 + 50 + 20]}{300}N$$
	$$\Rightarrow N_{sol^{\underline{n}}} = \frac{80}{300}N \approxeq \underline{0.27N}$$
	
	
	
	
\end{document}